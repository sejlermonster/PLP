%%%%%%%%%%%%%%%%%%%%%%%%%%%%%%%%%%%%%%%%%%%%%%%%%%%%%%%%%%%%%%%%%%%%%%%%%%%%%%%%
%                       Custom for figures (myFigure etc.)                     %
%                     			      	        		                       %
%%%%%%%%%%%%%%%%%%%%%%%%%%%%%%%%%%%%%%%%%%%%%%%%%%%%%%%%%%%%%%%%%%%%%%%%%%%%%%%%

% Include image and graphic
\usepackage{graphicx}
\usepackage{caption}
\usepackage{subcaption}

% Package for wrapping figures in text
\usepackage{wrapfig}

% Set the path where LaTeX looks for pictures.
\graphicspath{{figure/}}

% You need a newsubfloat element to use subcaption
\newsubfloat{figure}

% Command to set caption styles
\captionsetup[figure]{
    labelfont=bf
}

\captionsetup[table]{
    labelfont=bf
}

% from the old preamble
\captionnamefont{\bfseries\small}
\captiontitlefont{\itshape\small}
\subcaptionlabelfont{\bfseries\small}
\subcaptionfont{\itshape\small}


% Command \myFigure{filename}{caption}{label}{width} 
% for inserting a new figure.
\newcommand{\myFigure}[4]
{ 
    \begin{figure}[ht] 
        \centering 
        \includegraphics[width=#4\textwidth]{#1} 
        \caption{#2} 
        \label{#3} 
    \end{figure}
} 
 
% Insert a figure wraped in text (left/right)
% Command \myWrapFigure{filename}{caption}{label}{width}{r/l}
\newcommand{\myWrapFigure}[5]
{ 
    \begin{wrapfigure}{#5}{#4 \textwidth}
        \begin{center}
            \includegraphics[width=#4\textwidth]{#1}
        \end{center}
        \caption{#2}
        \label{#3} 
    \end{wrapfigure}
}
 
% Insert two figures side by side, with a caption covering both and a subcaption for each figure. OBS: scaled for 50 % text width both
% Command \mySubFigure{filename1}{filename2}{caption}
% {subcaption1}{subcaption2}{label}{sublabel1}{sublabel2}
\newcommand{\mySubFigure}[9]
{
    \begin{figure}[ht]
        \centering
        \subbottom[{#4}\label{#7}]%
            {\includegraphics[width=0.5\textwidth]{#1}}\hfill
        \subbottom[{#5}\label{#8}]%
            {\includegraphics[width=0.5\textwidth]{#2}}
        \caption{#3}
        \label{#6}
    \end{figure}
}


% Keeps floats in the section. 
% Help control place to put pix
\usepackage[section]{placeins}


%Package with enumirate command with extra options (used within cells of tabular)
\usepackage[inline]{enumitem}