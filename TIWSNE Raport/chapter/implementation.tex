\chapter{Implementation}
\label{chp:impl}

This chapter explains the design and implementation of the application and the choices made during the development.

\section{Architecture}
The objective of the project is to send an image, compressed and uncompressed, from one mote to another. To do this two PCs and two TelosB motes has been used, one sender mote and one receiver mote. The architecture of the system is shown in Figure \ref{fig:SystemOverview}. \emph{PC1} has the responsibility of converting an image into a byte array and transferring it to Mote1 over serial communication. When it is transfered to \emph{Mote1} the implemented \emph{SerialCom} module is used. The image is stored in the flash memory, because it is to big to be stored in the RAM. From the flash it is compressed and sent to \emph{Mote2}, via the \emph{RadioCom} module.

\myFigure{classDiagram.png}{The architecture of the system}{fig:SystemOverview}{0.7}

As seen in the figure \ref{fig:SystemOverview} the same modules and interfaces are also used in \emph{Mote2}. The compressed data is received in \emph{Mote2}, then decompressed and stored in the flash. From the flash it is send to \emph{PC2} via serial communication. 

%The image which is send around is only the raster of the original BMP.
%Be the image is transferred the header and footer has been stripped. When a PC receives the image back from a TelosB the image is then reconstructed according to its original.

\section{Serial communication}
%The serial communication is used to  the image from the PC to the TelosB mote. when the image has been passed, it will be stored in the external flash memory of the TelosB. When the image has been transferred between the TelosB's the serial communication was used to send the image back to the PC. On the PC the image is reconstructed and the image is saved on the PC.

%To enable serial communication on the TelosB an interface for serial communication was created. This interface gave access to the serial communication features. 

%Before the image was transferred from either the PC or the TelosB the image was divided into packages of 64 bytes. The intended receiver kept waiting for packages until the sender signals it is the last package the receiver is receiving. Then the receiver will stop the process of receiving packages. 

%Everytime a package is received on a TelosB the 64 bytes of the image is written to the flash. When a package is received on a PC the 64 bytes of the image is stored in an array. 

The serial communication is used to transfer the image between the PCs and the TelosB motes. 

In figure \ref{fig:serialCom} the serial communication between \emph{PC1} and \emph{Mote1} is explained. \emph{PC1} transfers a stream of packages to \emph{Mote1}. The image is transferred in packages of 64 bytes and \emph{Mote1} writes the data to the flash when it is received. When the last packages is signaled the flash is synchronized.

\myFigure{serialCom.png}{Serial Communication from \emph{PC1} to \emph{Mote1}}{fig:serialCom}{0.7}

Similar when the sink mote transfers the image to \emph{PC2} the PC reconstruct the image when the last package is received.

\section{Flash storage}

When a TelosB mote receives the image the flash storage is required as the used image has the size of 65,536 bytes. To fulfill this requirement the Block storage is used because it can store large objects.

The Block storage is a write-once model of storage. Rewriting requires an erase. Rewriting is required in this project as the image needs to be transferred with different compressions and therefore new data needs to be written to the flash.

To be able to use the Block storage the flash chip needs to be divided into one or more fixed-sized volumes. The volume is specified at compile-time. The volume size used in this project is 65,536 bytes which is the maximum size.

The interface \emph{Flash} was created to accommodate \emph{Mote1} and \emph{Mote2} which both needs to be able to write and read from the flash. \emph{Flash} is presented in Listing \ref{lst:flash}.

\begin{lstlisting}[caption={The interface flash}, label=lst:flash]
interface Flash
{
	command error_t write(uint8_t* data, uint32_t addr, uint8_t size);
	command error_t read(uint8_t* data, uint32_t addr, uint8_t size);
	command error_t sync();
	command error_t erase();
	event void readDone(error_t result, storage_len_t len);
	event void writeDone(error_t result);
	event void syncDone(error_t result);
	event void eraseDone(error_t result);	
}
\end{lstlisting}

The implementation of write requires three parameters. The first parameter is the data which should be written to the flash. The second parameter is the start address where the data should be written to the flash. The third parameter is the length of the data. 

The implementation of \emph{read} is similar to the \emph{write} command. Instead the data pointer is the returned data read from the flash.

Other commands like \emph{sync} and \emph{erase} are also implemented. The command \emph{sync} is used to finalize writes to the volume. The \emph{sync} command must be issued to ensure the data is stored in a non-volatile storage. 

The interface also includes events for all the commands. 

\section{Image Compression}
Three different lossy image compressions has been implemented in this project. A 1-bit, 2-bit and a 4-bit compression. Each of the compressions compresses the image by removing n-bits from every byte. 

The 1-bit compression is illustrated in figure \ref{fig:1bitCom}. For every 8 bytes the bits of the 8th byte replaces the least significant bit(LSB) of the 7 other bytes. This is done by performing an AND operation on byte 1-7 with the value (11111110). This AND operation leaves bit 1-7 untouched and sets the LSB to 0.  Afterwards the bits of the 8th byte is added to byte 1-7 by using an OR operation.

\myFigure{1BitCompression.png}{1 bit Compression}{fig:1bitCom}{0.92}
\FloatBarrier

The 1-bit decompression is illustrated in figure \ref{fig:1bitDecom}. For every 7 bytes the LSB in each is used to create the 8th byte. The LSB of all 8 bytes is then set to 0.

\myFigure{1BitDecompression.png}{1 bit decompression}{fig:1bitDecom}{1}


The implementation of the 2-bit and 4-bit compression is created similar to the 1-bit compression. The 2-bit compression maps 4 bytes into 3 bytes and the 4-bit compression maps 2 bytes into 1 byte.
\FloatBarrier
All implemented compressions and decompressions are done using bitwise operators on the array of bytes. Furthermore the implemented compressions and decompressions does not require more computation for a higher compression ratio.


\section{Radio communication}
Radio communication was used for transferring an image from one mote to another. The transferred image is read from the flash in chunks of 1024 bytes and saved to a buffer. Afterwards it is transferred to \emph{Mote2} in packages of 64 bytes. Figure \ref{fig:radioCom} presents the transfer between \emph{Mote1} and the sink, \emph{Mote2}.

%Before each transfer the \emph{Sender} side, the image has been reed in chunks of 1024 bytes. The chunk of 1024 bytes is loaded into a buffer. The buffer is then divided into packages of 64 bytes and then sent to the receiving TelosB. The sequence diagram in Figure \ref{fig:radioCom} gives an overview of how this process works.


\myFigure{MoteToMoteCom}{SysML sd: Mote to Mote radio communication}{fig:radioCom}{0.9}



The sequence diagram in figure \ref{fig:radioCom} illustrates how a non-compressed image is transferred. \emph{Mote1} start by reading 1024 bytes from the flash when the user button is pressed. It then transfers 16 packages and waits for an acknowledge. When \emph{Mote1} has received the acknowledge it will read another 1024 bytes from the flash and transfer the next 16 packages. When transferring the last package the status will be set to \emph{TRANSFER\_DONE} and the stream of bytes is transferred.

\FloatBarrier

%When the \emph{Sender} has sent 16 packages with 64 bytes in each. Then the \emph{Sender} waits for a \emph{TRANSFER\_ACK } from the \emph{Receiver}. The \emph{Recevier} sends the \emph{TRANSFER\_ACK } to signal that the \emph{Receiver} is ready to receive 16 new packages. When the \emph{Sender} receives the \emph{TRANSFER\_ACK } the \emph{Sender} will read the next chunk of 1024 bytes and put it into the buffer.

%When the last package of the image is about to be send the \emph{Sender} sends a \emph{TRANSFER\_DONE} with the last package. The \emph{Receiver} will not send any \emph{TRANSFER\_ACK} back and the \emph{Sender} will neither expect a \emph{TRANSFER\_ACK }.

Each time \emph{Mote2} receives a package of 64 bytes it stores the data in a buffer. The buffer has the size of 1024 bytes. When the buffer is full it is written to the flash. If a compression is applied the data in the buffer will be decompressed before writing it to the flash. Similar \emph{Mote1} will compress after reading from the flash and before transferring.

If the image has been compressed it will transfer the packages corresponding to the compressed size. For example, if the image has been compressed with a 4-bit compression the 1024 bytes will be reduced to 512 bytes. So instead of transferring 16 packages the \emph{Mote1} transfers 8 packages of 64 bytes each and waits for an acknowledge.

If the image is being compressed, \emph{Mote1} signals which compression has been used by adding the information to the header. \emph{Mote2} then  decompresses based on the header information.