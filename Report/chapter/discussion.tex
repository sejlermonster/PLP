\chapter{Discussion}
\label{chp:disc}

In this chapter the implementation and test decisions will be discussed. 

\section{Scheme}
In the application, Bresenham’s line algorithm was used for drawing a line. This algorithm is not able to draw vertical lines only slopes. This was solved by adding functionality to the algorithm so it could draw vertical lines. In the first implementation of the algorithm it could not handle lines where the x value of the second coordinate was greater than the x value in the first coordinate. This was solve by checking the two values, and swapping them, if the x value of the second coordinate was greatest.

In the beginning all functions were implemented with classic recursion. This led to some functions causing stack overflow. To resolve this, tail recursion was used. The reason for this, is that tail recursion allows the compiler to optimize and avoid allocation of a new stack frame for each recursive call. 

It was chosen not to keep the bounding-box state in Scheme, because it would violate the basic principles of functional programming. By not saving the state of the bounding-box in Scheme it was ensured that the Scheme code had no side effects. Instead the state of the bounding box was saved in C\#. An alternative option was to compromise and use the \emph{set!} operator to save the state of the bounding-box in Scheme.

\section{C\# WPF}
For rendering the coordinates which the scheme code yielded a canvas was used at first. The canvas worked well together with the MVVM structure of the WPF application. When the fill functionality was added the speed of the rendering onto the canvas was not satisfactory. A decision was made to use a bitmap instead and examine the speed of it. The result revealed that bitmap was fastest at drawing pixels. Therefore the bitmap replaced the canvas. 

\section{Test}

When implementing the unit tests it was important to ensure all boundary cases were tested. When unit testing the line function in Scheme eight unit tests were implemented to test different boundary cases. An example of the chosen boundary cases are shown on figure \ref{fig:direction}. Errors in the implementation was found in the line function when testing all boundaries. The unit tests could be considered a success as they helped identify bugs and made sure the source code had the expected behavior.

\myFigure{direction.png}{Testing line in multiple directions}{fig:direction}{0.27} 
\FloatBarrier

It was likewise a possibility to do an integration test of the complete system but it was chosen not to do this as the complete system was quite small. Integrations tests could have been created in C\# by using the concrete implementation instead of mocks.