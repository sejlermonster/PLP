\chapter{Test}
\label{chp:test}

Testing is an important part of a project. It is used to check the behavior of the source code, and to make sure that the functionality is correct.

In this project testing is divided into two parts. The testing of the Scheme functions and the testing of the C\# WPF code.

\section{Scheme test}
The test of the drawing engine in scheme  is implemented i the test.ss file. For each function i Scheme.ss, \emph{line, rectangle, circle, fill, bounding-box, text-at} and \emph{draw} multiple tests are made. To test functions in scheme.ss from test.ss the scheme.ss file is loaded into the test file. A function in scheme is tested by comparing the output of a function to the expected output value. To do this, \emph{equal?} is used. An example is shown in listing \ref{equal}.

\begin{lstlisting}[caption={Comparison of actual output with the expected output with equal?}, label=equal]
(and (equal? (candidate 0 0 8 8) '(0 0 1 1 2 2 3 3 4 4 5 5 6 6 7 7 8 8)))
\end{lstlisting}

This example is taken from the \emph{test-line} function. The line function takes 4 parameters, \emph{x, y, x2} and \emph{y2}. The test feeds the parameters 0, 0, 8 and 8, and compares the result to see if it is equal to what is expected. The expected value is in this example \emph{'0 0 1 1 2 2 3 3 4 4 5 5 6 6 7 7 8 8'} which is a line from the point (0,0) to (8,8). The \emph{TestRun} function is used to executes the test functions. Listing \ref{testrun} shows how \emph{TestRunLine} passes the line function as a parameter to \emph{test-line}. 

\begin{lstlisting}[caption={TestRunLine}, label=testrun]
(define TestRunLine (test-line line))
\end{lstlisting}

When \emph{TestRunLine} is executed it will return a \emph{\#t} for true or a \emph{\#f} for false depending on the result of the assertion.

\section{C\# Unit Testing}
The C\# WPF application has been tested by using xUnit.net and ReSharper xunit test runner. The tests are structured by \emph{Arrange, Act} and \emph{Assert}. The tests are placed in a separate test project called GraphikosTests. Every method is tested for all relevant scenarios.

The framework Moq is used to mock out dependencies. This ensures unit testing instead of integration testing.

Unit testing makes the maintenance of the code less error prone. This could result in a faster development time.

To be able to test the bitmaps the drawn bitmap is converted into a byte array and compared with an expected bitmap, in the form of a byte array. If they do not match, the test fails. This testing method was chosen because bitmaps do not have any relevant properties to assert on.