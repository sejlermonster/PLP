\chapter{Implementation} 
\label{chp:impl}

This chapter explains the design and implementation of the application and the choices made during the development.

\section{Multi paradigm}
The application uses the functional programming language Scheme together with the imperative language C\#. This is done by using \emph{IronScheme} which is an implementation of Scheme that targets .NET. This allows to call Scheme functions directly from C\# code and hence combine the use of the two languages.

\section{User interface in C\# WPF}
For developing the graphical user interface C\# and WPF was used. This made it possible to create the required interface that allowed the user to draw on a bitmap.

This was done using Model-View-ViewModel(MVVM) architecture to separate the graphical user interface from the business logic. This ensured that the business logic was very loosely coupled with the view through data-binding and therefore easier to test. The is illustrated in figure \ref{fig:mvvm}

\myFigure{mvvm.png}{Mode-View-ViewModel}{fig:mvvm}{1} 

A class diagram shown on figure \ref{fig:class} illustrates the essential design and classes used. The ShellView presents the content for the user and is updated through data-binding by the ShellViewModel. ShellViewModel depends on the two interfaces IShemeHandler and IBitmapDrawing. By only depending on interfaces testability is ensured as mocks can be used instead. 

The SchemeHandler is a class that handles the evaluation of the scheme code by using IronScheme. BitmapDrawing is used for drawing on the bitmap. It allows for both drawing single pixels and text.

\myFigure{class.png}{UML class diagram}{fig:class}{0.6} 

\section{Drawing Engine in Scheme}
Scheme was used for developing the drawing engine. This allowed for fast and stateless computation of the graphics being shown in the user interface. 

For solving the different drawing aspects both classic and tail recursion was used. Tail recursion proved to have better performance and by using tail recursion instead of classic recursion stack overflow could be avoided. This is a because the scheme compiler is able to very effectively optimize tail recursion.

High order programming was used by passing functions to other functions. This was for example used whit the \emph{draw} function. The function was designed as a variadic function that takes a variable amount of other functions an combining their result. This was also used for the \emph{fill} 

