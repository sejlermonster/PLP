\chapter{Implementation} 
\label{chp:impl}

This chapter explain the implementation of the application.

\section{Multi paradigm}
The application uses the functional programming language Scheme together with the imperative language C\#. This is done by using \emph{IronScheme} which is an implementation of Scheme that targets .NET. This allows to call Scheme functions directly from C\# code and hence combine the use of the two languages.

This allowed... made it possible... something

\section{User interface in C\# WPF}
For developing the graphical user interface C\# and WPF was used. This made it possible to create the required interface that allowed the user to call scheme functions and have the objects drawn. 

Model-View-ViewModel(MVVM) was used to separate the graphical user interface from the business logic.  This ensured that the busines logic was very loosely coupled with the view through data-binding and therefore easier to test. The is illustrated in figure \ref{fig:mvvm}

\myFigure{mvvm.png}{Mode-View-ViewModel}{fig:mvvm}{1} 

Furthermore a bootstrapper was used for constructor injection. All classes that needs to be constructor injected was given a appropriate interface to ensure testability.

A bitmap was used to draw on. This made it possible to draw single pixels that was returned from the scheme functions.


Holder vores states
WPF til GUI
Tegner på Bitmap
Interfaces, bootstrapper, MVVM for at sikre testability

\section{Drawing Engine in Scheme}
Scheme was used for developing the drawing engine. This allowed for fast and stateless computation of the graphics being shown in the user interface. 

stateless

funktioner

high order programming ( Passing functions) bla bla